% Copyright (c) 2018 Robert Carnell

\documentclass[11pt,letter]{book}

\usepackage[T1]{fontenc}

% Fonts added for the book
\usepackage[bitstream-charter]{mathdesign} % use the bitstream-charter True Type font

% ---- Gramps Packages ----
%\usepackage[latin1]{inputenc}%
\usepackage[latin1,utf8]{inputenc}%
\usepackage{graphicx}% Extended graphics support
\usepackage{longtable}% For multi-page tables
\usepackage{calc}% For some calculations
\usepackage{ifthen}% For table width calculations
\usepackage{ragged2e}% For left aligning with hyphenation
\usepackage{wrapfig}% wrap pictures in text

% Packages added for the book
\usepackage[all]{genealogytree} % genealogy-tree package for trees
\usepackage{hyperref} % to break long URL lines
\usepackage{endnotes} % create endnotes
\usepackage{etoolbox} % modify endnotes
% endnote modifications
\renewcommand{\notesname}{Endnotes}
\patchcmd{\theendnotes}
  {\makeatletter}
  {\makeatletter\renewcommand\makeenmark{\theenmark.\space}}
  {}{}
\usepackage{draftwatermark} % create draft watermark
% watermark modifications
\SetWatermarkText{DRAFT}
\SetWatermarkScale{1}
\SetWatermarkColor[rgb]{0.9,0.9,0.9}

% Begin document
\title{\bf Carnell-Rettgers Family Genealogy \\
       \large Volume I - Military Service}
\author{Robert Carnell}
\date{\today}

\begin{document}
\frontmatter
\maketitle
\clearpage

\begingroup
\footnotesize
\parindent 0pt
\parskip
\baselineskip
\textcopyright{} 2018 Robert Carnell
\endgroup
\clearpage

\normalsize

\chapter{Preface}

I started the journey to document our family history in 2014 after both my grandfathers had passed away, Curtis T Carnell (1922-1991) and Robert E Rettgers (1923-2014).  They were both part of the Greatest Generation, both volunteered before Pearl Harbor, and both served their country to the utmost of their ability in the Pacific.  Neither one talked much about their military service.  Grandpa Curtis said that the Japanese were coming, but never got to him.  Grandpa Bob talked about the interesting mechanical and electronic aspects of destroyer gunnery, some about ship board life, and about one kamikaze attack on his ship.  Neither one talked about the terrible aspects of war that they experienced.  I will always regret not having asked more questions and not having said thank-you for what they did.  I only hope that I am honoring them with this effort.  

I never would have expected to find these incredible people in my family.  Immigrants, pioneers, farmers, and heroes fill these pages.  We don't talk enough about our family history.  This history largely covers much of the history of the United States as well.  Everyone in this family should be proud.

I have been helped along the way by many of family members.  My mother and father have been a tremendous help in collecting the family's knowledge, visiting relatives, and documenting graves from Georgia to Pennsylvania.  I also made use of other's work in the beginning of the project.  My Aunt, Sharon Rettgers Idema, had done quite a bit of research at the Family History Center on the Rettgers / Fett family and graciously provided it all.  Cousin, Shirley Quackebush Frye (1950-2002) had done extensive research on the Smith / Josey branch all before the internet made this much easier.  Finally, Janice McMillan has been my window into learning about the Sons and Daughters of the American Revolution and about how to be a careful amateur genealogist.

If you come across any errors or omissions in these pages, please contact me.  I will ask that you provide references or documents if possible.  As you can see from the extensive footnotes in these pages, I make every effort to include original sources when possible.

\tableofcontents
\mainmatter
\chapter{Introduction}

The descendants of the people contained in these volumes are related to a long line of military heroes.  Documented here are the direct ancestors of William Connell, Nellie Langston, Kendrick Smith, Jewel Josey, Ithamar Rettgers, Laura Hartenstine, Floyd Fett, and Florence Becthel with military service.  Their children with military service are also documented.  For additional information on great aunts and uncles and other more recent military service, please see \url{https://militaryancestry.wordpress.com/}.

President John F. Kennedy gave remarks at the U.S. Naval Academy on August 1, 1963.  His words (in bold below) are engraved on the Navy Memorial in Washington, D.C.;  they apply to all those who have served:

\begin{quotation}
...I want to express our very strong appreciation to all those of you in the plebe class who have come into the Navy. I hope that you realize how great is the dependence of our country upon the men who serve in our Armed Forces. I sometimes think that the people of this country do not appreciate how secure we are because of the devotion of the men and their wives and children who serve this country in far off places, in the sea, in the air, and on the ground, thousands and thousands of miles away from this country, who make it possible for us all to live in peace each day.

This country owes the greatest debt to our servicemen. In time of war, of course, there is a tremendous enthusiasm and outburst of popular feeling about those who fight and lead our wars, but it is sometimes different in peace. But I can assure the people of this country, from my own personal experience in the last 21/2 years, that more than anything, more than anything, the fact that this country is secure and at peace, the fact that dozens of countries allied with us are free and at peace, has been due to the military strength of the United States. And that strength has been directly due to the men who serve in our Armed Forces. So even though it may be at peace, in fact most especially because it is at peace, I take this opportunity to express our appreciation to all of them whether they are here at Annapolis, or whether they are out of sight of land, or underneath the sea.

I want to express our strong hope that all of you who have come to the Academy as plebes will stay with the Navy. I can think of no more rewarding a career. You will have a chance in the next 10, 20, and 30 years to serve the cause of freedom and your country all over the globe, to hold positions of the highest responsibility, to recognize that upon your good judgment in many cases may well rest not only the well-being of the men with whom you serve, but also in a very real sense the security of your country.

\textbf{I can imagine a no more rewarding career. And any man who may be asked in this century what he did to make his life worth while, I think can respond with a good deal of pride and satisfaction: ``I served in the United States Navy.''} 

So I congratulate you all. This is a hard job, particularly now as you make the change, but I think it develops in you those qualities which we like to see in our country, which we take pride in. I am sure you are going to stay with it. I am sure you are going to be able, by what you are now going through, to find the means to command others.

So I express our very best wishes to you and tell you that though you will be serving in the Navy in the days when most of those who hold public office have long gone from it, I can assure you in 1963 that your services are needed, that your opportunities are unlimited, and that if I were a young man in 1963 I can imagine no place to be better than right here at this Academy, or at West Point, or in the Air Force, or in some other place beginning a career of service to the United States.
\end{quotation}

%%------------------------------------------------------------------------------
\chapter{Revolutionary War}

\section{Patriots With Military Service}

\begin{enumerate}
\item Christian Bechtel
\begin{itemize}
\item 1752-1814
\item Lieutenant in Capt Joseph Sands Company, 5th Battalion Berks County Militia (Pennsylvania) \endnote{PA Archives, Series 3, Volume VI, pg 288, Berks County Accounts of Lieutenants}
\item Lieutenant in Capt Weidner's Company in Col Lutz's 1st Battalion of Berk's County in Apr and May 1785 \endnote{PA Archives, Series 6, Volume III, pg 55, Militia Rolls of Berk’s County 1783-1790}
\item Captain of the 6th Company of the 6th Regiment \endnote{PA Archives, Series 6, Volume 5, pg 129, Pennsylvania Militia 1790-1800, Berks County, Muster and Pay Rolls}
\end{itemize}

\item Daniel Deturk
\begin{itemize}
\item 1742-1791
\item Company commander, Capt, 1st Battalion Berks County Militia, Lt Col Haller \endnote{PA Archives, 5th Series, Volume 5, page 138}
\item 3rd Company Commander, Capt, 4th Battalion Berk County Militia, Col Nicholas Lutz \endnote{PA Archives, 5th Series, Volume 5, page 203}
\item Captain, 6th Battalion, Berks County Militia, Col Spycker \endnote{PA Archives, 5th Series, Volume 5, page 251}
\end{itemize}

\item Michael Zerbe
\begin{itemize}
\item 1744-1806
\item Sergeant
\item Capt George Miller's Company, Berks County Militia \endnote{PA Archives, 2nd Series, Volume 14, page 250}
\end{itemize}

\item John Van Reed
\begin{itemize}
\item 1747-1820
\item Capt Sebastian Miller's Company, Col Joseph Hiester, Berks County Militia \endnote{PA Archives, 5th series, Volume 5, pg 211}
\item Pvt
\end{itemize}

\item Isaac High
\begin{itemize}
\item 1753-1795
\item Capt Heinrich Nach's Company, Berks County Militia \endnote{PA Archives, 5th series, Volume 5, pg 188}
\end{itemize}

\item John Reese
\begin{itemize}
\item 1754-1813
\item Capt Geist and Capt Whetstone's Companies, Col Hunter, Berks County Militia \endnote{Pension R17339V}
\end{itemize}

\item William Gainer
\begin{itemize}
\item 1758-1800
\item Capt Benjamin Spiller and Augustine Tabb, 2nd Virginia State Regiment \endnote{National Archives, M881, Compiled Military Service Records, Roll 947}
\begin{itemize}
\item Note: these are filed under William Ganer and William Garner
\end{itemize}
\item Valley Forge in Spring of 1778 \endnote{Valley Forge Muster Roll Project, ID VA10248}
\begin{itemize}
\item Note: Here he is listed as William Gainer 
\end{itemize}
\end{itemize}

\item Lewis Jenkins
\begin{itemize}
\item abt 1760-
\end{itemize}

\item Abner Broach
\begin{itemize}
\item abt 1761-1810
\item Continental Line, Enlisted at Chesterfield, Virginia \endnote{Virginia Society Quartertly, volume 2, pg 50},\endnote{Size Roll of the Troops Joined at Chesterfield County House since 1st Sep 1780}
\end{itemize}

\item Benjamin Smith
\begin{itemize}
\item - 1799
\item Paid for services \endnote{North Carolina Revolutionary War Pay Vouchers 2914, 2867, 2437, 14, ROLL S.115.125}
\end{itemize}

\item William Hottenstein
\begin{itemize}
\item 1730-1784
\item Court Martial Man
\item Col Nicholas Lotz, 4th Battalion, Berks County Militia and Capt George Reehm
\end{itemize}

\section{Designated Patriots Without Verified Military Service}

\item John Deturk
\begin{itemize}
\item 1713-
\item Took Oath of Allegiance, 1778
\end{itemize}

\item Henry Van Reed
\begin{itemize}
\item 1722-1790
\item Signed Oath of Allegiance, 1777
\end{itemize}
\end{enumerate}

%%------------------------------------------------------------------------------
\chapter{War of 1812}

\begin{enumerate}
\item William High
\begin{itemize}
\item 1786 - 1851
\item 1809, Reading Pennsylvania Calvary \endnote{Historical and Biographical annals of Berks County, Pennsylvania, pg 785}
\item 1816, Captain, Reading Calvary \endnote{Historical and Biographical annals of Berks County, Pennsylvania, pg 785}
\item 1841, Brigadier General, 2nd Brigade, 6th Division, Pennsylvania Militia \endnote{Description of the Borough of Reading, pg 61}
\end{itemize}

\item Christian Bechtel
\begin{itemize}
\item 1786-1839
\item Findlay's Battalion, Pennsylvania Volunteers \endnote{War of 1812 Service Records, National Archives}
\item Pvt
\end{itemize}
\end{enumerate}

%%------------------------------------------------------------------------------

\chapter{Civil War}

\section{Union}

\begin{enumerate}
\item William Joseph Ritz
\begin{itemize}
\item 1813 - 1880
\item Company D, 167th Pennsylvania Infantry Regiment
\item (12 Nov 1862 - 12 Aug 1863 (sick))
\item Pvt
\end{itemize}

\item Heber Hartenstine
\begin{itemize}
\item 1838 - 1879
\item Company B, 175th Pennsylvania Infantry Regiment
\item Pvt
\end{itemize}

\section{Confederate}

\item William Jordan Smith (fold3)
\begin{itemize}
\item 1836 - 1922
\item Company E, 48th Georgia Infantry Regiment
\item Enlisted:  4 Mar 1862
\item Wounded:  17 Sep 1862 at Sharpsburg, MD
\item Paroled:  24 May 1865 at Augusta, GA
\item 2nd Lt, 1st Lt, Capt
\end{itemize}

\item Thomas M McNeely (fold3)
\begin{itemize}
\item 1843 - 1908
\item Company E, 48th Georgia Infantry Regiment
\item Pvt
\end{itemize}

\item Elias Langston
\begin{itemize}
\item 1820 - 1863
\item Company G, 1st (Butler's) South Carolina Infantry
\item Pvt
\item Died at Fort Moultrie, Sullivan's Island, SC
\end{itemize}

\item James E Broach
\begin{itemize}
\item 1825 - 1899
\item Company G, 26th South Carolina Volunteer Infantry Regiment
\item Pvt, Cpl
\end{itemize}

\item Richard M Nunnery
\begin{itemize}
\item 1844 - 1896
\item Company G, Hampton Legion, South Carolina Infantry Regiment
\item Pvt
\end{itemize}

\item Everett Jackson Josey
\begin{itemize}
\item 1847 - 1914
\item Company C, 5th Georgia Reserve Infantry Regiment
\item Pvt
\end{itemize}

\item Henry Addison Josey
\begin{itemize}
\item 1827 - 1890
\item Citizens mustered for service in Washington County, GA, but never activated.  This unit was identified as the 13th Regiment, Georgia Militia
\end{itemize}

\end{enumerate}

%%------------------------------------------------------------------------------

\chapter{Spanish American War}

\begin{enumerate}
\item Frederick H Rettgers
\begin{itemize}
\item 1873 - 1944
\item Company A, 4th Regiment, Pennsylvania Volunteers
\item Roster - See Richards, Fred
\item Entered Service:  9 May 1898
\item Honorable Discharge: 16 Nov 1898
\item Pvt
\end{itemize}
\end{enumerate}

%%------------------------------------------------------------------------------

\chapter{World War I}

\begin{enumerate}

\item William Buck Connell
\begin{itemize}
\item 1890 - 1953
\item Service
\begin{itemize}
\item Enlisted: 15 Jul 1918, Company D, 57th Pioneer Infantry
\item 29 July 1918, Company B, 53rd Pioneer Infantry
\item 6 Aug 1918, Deployed overseas with American Expeditionary Forces
\item 3 May 1919, Returned from overseas
\item Discharged:  17 May 1919
\end{itemize}
\item Pvt
\end{itemize}

\item Floyd William Fett
\begin{itemize}
\item 1891 - 1956
\item Service
\begin{itemize}
\item Enlisted 20 Dec 1917
\item 20 Dec 1917 - 6 Jul 1919, Company A, 315th Infantry
\item 9 Jul 1918 - 30 May 1919, Europe
\item 30 Aug 1918 - 11 Nov 1918, 1st Army Defensive Sector
\item Meuse-Argonne Campaign
\item Honorably Discharged:  6 Jul 1919
\end{itemize}
\item Ranks:
\begin{itemize}
\item Pvt
\item Corporal, 17 Jun 1918
\end{itemize}
\end{itemize}

\item Ithamar Benedict Rettgers
\begin{itemize}
\item 1898 - 1937
\item Service
\begin{itemize}
\item Enlisted 9 Apr 1917 in Pennsylvania National Guard
\item 9 Apr 1917 - 5 Aug 1917, Company I, 4th Infantry, Pennsylvania National Guard
\item Company I, 149th Machine Gun Battalion
\item Company A, 149th Machine Gun Battalion
\item Company D, 150th Machine Gun Battalion (roster)
\item 14 Nov 1917 - 28 Apr 1919, Overseas
\item Luneville sector, Lorraine, France, 21 Feb-23 Mar, 1918
\item Baccarat sector, Lorraine, France, 31 Mar-21 Jun, 1918
\item Esperance-Souain sector, Champagne, France, 4 Jul-14 Jul, 1918
\item Champagne-Marne defensive, France, 15 Jul-17 Jul, 1918
\item Aisne-Marne offensive, France, 25 Jul-3 Aug, 1918
\item St. Mihiel offensive, France, 12 Sep-16 Sep, 1918
\item Essey and Pannes sector, Woevre, France, 17 Sep -30 Sep 1918
\item Meuse-Argonne offensive, France, 12 Oct-31 Oct, 1918
\item Meuse-Argonne offensive, France, 5 Nov-10 Nov, 1918
\item Army of Occupation
\item Discharged 7 May 1919 back to National Guard
\end{itemize}
\item Ranks
\begin{itemize}
\item Cpl
\item Sgt 24 Apr 1918
\end{itemize}
\item Note: He was not wounded, but was gased
\end{itemize}

\end{enumerate}

%%------------------------------------------------------------------------------

\chapter{World War II}

\begin{enumerate}

\item Robert Rettgers
\begin{itemize}
\item 1923 - 2014
\item Service:
\begin{itemize}
\item Enlisted, 27 Aug 1941
\item Boot Camp, Naval Training Station, Newport, RI
\item 23 Oct 1941 - 28 Aug 1941
\item US Naval Training School, San Diego, CA
\item 28 Oct 1941 - 25 Feb 1942
\item USS Helena (CL-50)
\item 3 Mar 1942 - 23 May 1942
\item Photo of USS Helena, 10 days after arrival (13 Mar 1942)
\item US Naval Hospital, Mare Island, CA
\item 23 May 1942 - 12 Aug 1942
\item USS Saufley (DD-465)
\item 1 Sep 1942 - 22 Jun 1945
\item Advanced Fire Control School, Washington, D.C. Navy Yard
\item 9 Aug 1945 - 31 Jan 1946
\item USS Boston (CA-69)
\item 19 Feb 1946 - 28 Aug 1946
\item Honorable Discharge, 28 Aug 1946
\end{itemize}
\item Ranks
\begin{itemize}
\item S3/C (27 Aug 1941)
\item S2/C (~3 Mar 1942)
\item S1/C (1 May 1942)
\item FC3/C (6 Nov 1942)
\item FC2/C (11 May 1943)
\item FC1/C (1 Apr 1944)
\item Warrant Officer - Gunner (15 Aug 1945)
\end{itemize}
\end{itemize}

\item Curtis Carnell
\begin{itemize}
\item 1922-1991
\item Service:
\begin{itemize}
\item Enlisted, 20 Sep 1940
\item Company M, 34th Infantry Regiment, 24th Infantry Division
\item Army's famous "Follow Me" poster -
\item Honorable Discharge, 28 Jun 1945
\end{itemize}
\item Ranks
\begin{itemize}
\item Sgt
\end{itemize}
\end{itemize}

\item Wilbur Cornell
\begin{itemize}
\item 1920-2008
\item Service:
\begin{itemize}
\item Enlisted, 11 Aug 1944
\item 290th Infantry Regiment (Unconfirmed)
\item Discharge, 26 May 1946
\end{itemize}
\item Ranks:
\end{itemize}

\item Forrest Rettgers
\begin{itemize}
\item 1921 - 1995
\item Service:
\begin{itemize}
\item Pennsylvania National Guard
\item Entered Active Service:  16 Sep 1940
\item Pennsylvania National Guard, 213th Coast Artillery
\item 7 Dec 1941 - 11 Oct 1944, Domestic Service
\item 12 Oct 1944 - ?, Overseas Service
\item Office of Secretary of Defense
\item US Army War College
\item Senate Liaison Officer
\item 1966, Retired
\end{itemize}
\item Ranks
\begin{itemize}
\item Dec 1941, 2nd Lt
\item Oct 1942, 1st Lt
\item 1943, Capt
\item Mar 1953, Lt Col
\item Col
\end{itemize}
\item Buried in Arlington National Cemetery
\end{itemize}

\end{enumerate}

%%------------------------------------------------------------------------------
\newpage
\begingroup
\parindent 0pt
\parskip 2ex
\def\enotesize{\normalsize}
\theendnotes
\endgroup

\end{document}
