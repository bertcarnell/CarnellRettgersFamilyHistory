\documentclass[11pt,letter]{book}

\usepackage[T1]{fontenc}

% Fonts added for the book
\usepackage[bitstream-charter]{mathdesign} % use the bitstream-charter True Type font

% ---- Gramps Packages ----
%\usepackage[latin1]{inputenc}%
\usepackage[latin1,utf8]{inputenc}%
\usepackage{graphicx}% Extended graphics support
\usepackage{longtable}% For multi-page tables
\usepackage{calc}% For some calculations
\usepackage{ifthen}% For table width calculations
\usepackage{ragged2e}% For left aligning with hyphenation
\usepackage{wrapfig}% wrap pictures in text

% Packages added for the book
\usepackage[all]{genealogytree} % genealogy-tree package for trees
\usepackage{endnotes} % create endnotes
\usepackage{etoolbox} % modify endnotes
% endnote modifications
\renewcommand{\notesname}{Endnotes}
\patchcmd{\theendnotes}
  {\makeatletter}
  {\makeatletter\renewcommand\makeenmark{\theenmark.\space}}
  {}{}

% Begin document
\title{\bf Carnell-Rettgers Family Genealogy \\
       \large Volume I - Military Service}
\author{Robert Carnell}
\date{\today}

\begin{document}
\frontmatter
\maketitle
\chapter{Preface}

I started the journey to document our family history in 2014 after both my grandfathers had passed away, Curtis T Carnell (1922-1991) and Robert E Rettgers (1923-2014).  They were both part of the Greatest Generation, both volunteered before Pearl Harbor, and both served their country to the utmost of their ability in the Pacific.  Neither one talked much about their military service.  Grandpa Curtis said that the Japanese were coming, but never got to him.  Grandpa Bob talked about the interesting mechanical and electronic aspects of destroyer gunnery, some about ship board life, and about one kamikaze attack on his ship.  Neither one talked about the terrible aspects of war that they experienced.  I will always regret not having asked more questions and not having said thank-you for what they did.  I only hope that I am honoring them with this effort.  I never would have expected to find the incredible people in my family.  Immigrants, pioneers, farmers, and heroes fill these pages.  

I have been helped along the way by many of family members.  My mother and father have been a tremendous help in collecting the family's knowledge, visiting relatives, and documenting graves from Georgia to Pennsylvania.  I also made use of other's work in the beginning of the project.  My Aunt, Sharon Rettgers Idema, had done quite a bit of research at the Family History Center on the Rettgers / Fett family and graciously provided it all.  Cousin, Shirley Quackebush Frye (1950-2002) had done extensive research on the Smith / Josey branch all before the internet made this much easier.  Finally, Janice McMillan has been my window into learning about the Sons and Daughters of the American Revolution and about how to be a careful amateur genealogist.

If you come across any errors or omissions in these pages, please contact me.  I will ask that you provide references or documents if possible.  As you can see from the extensive footnotes in these pages, I make every effort to include original sources when possible.

\tableofcontents
\mainmatter
\chapter{Introduction}

The descendants of the people contained in these volumes are related to a long line of military heroes.  Documented here the direct ancestors with military of William Connell, Nellie Langston, Kendrick Smith, Jewel Josey, Ithamar Rettgers, Laura Hartenstine, Floyd Fett, and Florence Becthel.  Their children with service are also documented.  For additional information on great aunts and uncles and other more recent military service, please see https://militaryancestry.wordpress.com/.

%%------------------------------------------------------------------------------
\chapter{Revolutionary War}

\section{Patriots With Military Service}

\begin{enumerate}
\item Christian Bechtel
\begin{itemize}
\item 1752-1814
\item Lieutenant in Capt Joseph Sands Company, 5th Battalion Berks County Militia (Pennsylvania) \endnote{PA Archives, Series 3, Volume VI, pg 288, Berks County Accounts of Lieutenants}
\item Lieutenant in Capt Weidner's Company in Col Lutz's 1st Battalion of Berk's County in Apr and May 1785 \endnote{PA Archives, Series 6, Volume III, pg 55, Militia Rolls of Berk’s County 1783-1790}
\item Captain of the 6th Company of the 6th Regiment \endnote{PA Archives, Series 6, Volume 5, pg 129, Pennsylvania Militia 1790-1800, Berks County, Muster and Pay Rolls}
\end{itemize}

\item Daniel Deturk
\begin{itemize}
\item 1742-1791
\item Company commander, Capt, 1st Battalion Berks County Militia, Lt Col Haller \endnote{PA Archives, 5th Series, Volume 5, page 138}
\item 3rd Company Commander, Capt, 4th Battalion Berk County Militia, Col Nicholas Lutz \endnote{PA Archives, 5th Series, Volume 5, page 203}
\item Captain, 6th Battalion, Berks County Militia, Col Spycker \endnote{PA Archives, 5th Series, Volume 5, page 251}
\end{itemize}

\item Michael Zerbe
\begin{itemize}
\item 1744-1806
\item Sergeant
\item Capt George Miller's Company, Berks County Militia \endnote{PA Archives, 2nd Series, Volume 14, page 250}
\end{itemize}

\item John Van Reed
\begin{itemize}
\item 1747-1820
\end{itemize}

\item Isaac High
\begin{itemize}
\item 1750-1795
\end{itemize}

\item John Reese
\begin{itemize}
\item 1754-1813
\end{itemize}

\item William Gainer
\begin{itemize}
\item 1758-1800
\end{itemize}

\item Lewis Jenkins
\begin{itemize}
\item abt 1760-
\end{itemize}

\item Abner Broach
\begin{itemize}
\item abt 1761-1810
\end{itemize}

\item Benjamin Smith
\begin{itemize}
\item - 1799
\end{itemize}

\item William Hottenstein
\begin{itemize}
\item 1730-1784
\item Court Martial Man
\item Col Nicholas Lotz, 4th Battalion, Berks County Militia and Capt George Reehm
\end{itemize}

\section{Designated Patriots Without Verified Military Service}

\item John Deturk
\begin{itemize}
\item 1713-
\item Took Oath of Allegiance, 1778
\end{itemize}

\item Henry Van Reed
\begin{itemize}
\item 1722-1790
\item Signed Oath of Allegiance, 1777
\end{itemize}
\end{enumerate}

%%------------------------------------------------------------------------------
\chapter{War of 1812}

\begin{enumerate}
\item William High
\begin{itemize}
\item 1786 - 1851
\item 1809, Reading Pennsylvania Calvary \endnote{Historical and Biographical annals of Berks County, Pennsylvania, pg 785}
\item 1816, Captain, Reading Calvary \endnote{Historical and Biographical annals of Berks County, Pennsylvania, pg 785}
\item 1841, Brigadier General, 2nd Brigade, 6th Division, Pennsylvania Militia \endnote{Description of the Borough of Reading, pg 61}
\end{itemize}

\item Christian Bechtel
\begin{itemize}
\item 1786-1839
\item Findlay's Battalion, Pennsylvania Volunteers \endnote{War of 1812 Service Records, National Archives}
\item Pvt
\end{itemize}
\end{enumerate}

%%------------------------------------------------------------------------------

\chapter{Civil War}

\section{Union}



\section{Confederate}




%%------------------------------------------------------------------------------

\newpage
\begingroup
\parindent 0pt
\parskip 2ex
\def\enotesize{\normalsize}
\theendnotes
\endgroup

\end{document}
